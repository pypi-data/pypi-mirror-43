%-------------------------------------------------------------------------------
% \author Jan P Buchmann <jan.buchmann@sydney.edu.au>
% \copyright 2018 The University of Sydney
% \description
%-------------------------------------------------------------------------------

\documentclass[a4paper]{article}
\usepackage[T1]{fontenc}
\usepackage[utf8]{inputenc}
\usepackage{lmodern}
\usepackage[breaklinks, pdfborder={0 0 0}]{hyperref}
\usepackage{breakurl}
\usepackage{graphicx}
\usepackage{lscape}
\usepackage{rotating}
\usepackage{natbib}
\usepackage{listings}
\usepackage{jpb.lst}
\usepackage{subfig}
\usepackage{xspace}
\usepackage{caption}
\usepackage[tracking=true]{microtype}
\usepackage[english]{babel}
\newcommand{\entrezpy}{\texttt{Entrezpy}\xspace}
\title{\entrezpy:A Python library to dynamically interact with the
       NCBI Entrez databases}
\begin{document}
  \begin{minipage}{.75\textwidth}
      \textbf{\large{\entrezpy:A Python library to dynamically interact with the
             NCBI Entrez databases}}
    \vspace{\baselineskip}
    \setlength{\parskip}{\smallskipamount}
    \begin{center}
    Jan P. Buchmann and Edward C. Holmes
    \vspace{\baselineskip}
    \setlength{\parskip}{\smallskipamount}\\
    \small{Marie Bashir Institute for Infectious Diseases and Biosecurity, Charles Perkins Centre, School of
    Life and Environmental Sciences and Sydney Medical School, The University of Sydney,
    Sydney, NSW 2006, Australia}\\[2em]
    \small{2018-12-04}
    \end{center}

  \end{minipage}
\vspace{2\baselineskip}

\section{Abstract}
  \paragraph{Summary:} \entrezpy is a Python library that automates the querying
  and downloading of data from the Entrez databases at NCBI by interacting with
  E-Utilities. \entrezpy implements complex queries analogous to piping commands
  on UNIX systems, enabling it to implement interactions with all Entrez
  databases as part of an analysis pipeline, which can also be adjusted
  on-the-fly. Entrezpy’s design allows its easy extension and adjustment to the
  existing E-Utility functions without changes in the underlying mechanisms.

  \paragraph{Availability and Implementation:} \entrezpy is implemented in
  Python 3 ($\geq$3.6) and depends only on the standard library. Its source code
  is available at \url{https://gitlab.com/ncbipy/entrezpy.git} and is licensed
  under the LGPL.

  \paragraph{Contact:} \href{jan.buchmann@sydney.edu.au}{jan.buchmann@sydney.edu.au}
  \paragraph{Supplementary information:}
    \burl{https://gitlab.com/ncbipy/entrezpy/wikis/entrezpy }

\section{Introduction}
  The increasing availability of biological data has not only resulted in a
  multitude of genome sequence data, but also substantial increases in the
  amount of accompanying metadata, including phylogenies, sampling conditions,
  and gene ontologies. These data can also be divided into several parts: for
  example, genome sequences are stored as individual chromosomes, while
  transcriptional experiments are split into several sequencing runs according
  to the specific questions addressed. To consolidate this burgeoning
  information, sequence repositories have developed specific databases that
  store and link this wealth of data. The National Center for Biotechnology
  Information (NCBI) is one of the largest such repositories and both developed
  and maintains the Entrez databases that currently encompass 37 individual
  databases storing 2.1 billion records related to the life sciences
  \citep{NCBI2016}. NCBI offers two approaches to interact programmatically
  with its Entrez databases:

  \begin{description}
    \item[Entrez Direct] is a powerful Perl program allowing ad hoc access to
    the NCBI databases through a command line interface (\citep{Kans2018} ,
    Figure~\ref{subfig:esearch}). It can be incorporated into Bash scripts or
    invoked within pipelines  as an external command. However, it is not
    designed as a library and therefore difficult to integrate into analysis
    pipelines.

    \item[E-utilities] (\url{http://eutils.ncbi.nlm.nih.gov/}) are a set of
    tools that allow the user to query and retrieve NCBI data using specific
    Uniform Resource Identifiers (URIs). NCBI data can be accessed using an URI
    describing the function and its parameter, such as searching a database
    with a specific term.
  \end{description}


Herein, we present \entrezpy. To our knowledge this is the first Python library
to offer the same functionalities as Entrez Direct. Existing libraries, such as
Biopython \citep{Cock2009} or ETE 3 \citep{Huerta-Cepas2016}, offer either a
basic or a very narrow approach to interact with E-utilities. Biopython does
not handle whole queries, leaving the user to implement the logic to fetch
large requests, while ETE represents a library focusing only on phylogenetics.
In contrast, \entrezpy is specifically designed to interact with E-Utilities
and automatically configures itself to retrieve large data sets. Its design,
the smaller code base, and that it only depends on the Python standard library,
make \entrezpy easier to install, maintain, adjust and extend.

\entrezpy includes the helper class, termed Wally, to facilitate the creation
and execution of complex queries and reusing previously obtained results. This
allows the user to query the Entrez database and retrieve the corresponding
data as part of an analysis pipeline. \entrezpy is licensed under the GNU Lesser
General Public License and can be obtained from
\url{https://gitlab.com/ncbipy/entrezpy}. It is documented using Doxygen and a
wiki (\burl{https://gitlab.com/ncbipy/entrezpy/wikis/entrezpy}) which has more
complex usage examples.

\section{Implementation}
Data sets within an Entrez database are identified by their identification
number. The Entrez documentation refers to this number interchangeably as
either UID or ID. For the remainder of this article we will use the term UID to
refer to a data set identification number. UIDs are unique within an Entrez
database but not across Entrez databases.

\entrezpy is a library of Python classes implementing the specific steps
required to interact with the E- Utilities. Querying and downloading data via
the E-Utility is achieved by sending queries encoded as an URI for the specific
function and the corresponding parameters. For example, the following URI
searches the nucleotide database for viral genomes and returns the UIDs
detected:
\url{https://eutils.ncbi.nlm.nih.gov/entrez/eutils/esearch.fcgi?db=nucleotide&term=viruses[orgn]}.

The E-Utility returns a response describing the search result. This includes
the number of data sets recovered within the requested database and the UID
sequence identifiers. To fetch this data set, an E-Utility URI must be
assembled. The following E-Utility URI fetches the first four sequences from
the previous query in the FASTA format:
\sloppy{\url{https://eutils.ncbi.nlm.nih.gov/entrez/eutils/efetch.fcgi?db=nucleotide&id=1509580163,1509580026,1509580024,1509580022&rettype=fasta&retmode=text}}.

\entrezpy supports the E-Utilities EFetch, ESearch, ELink, ESummary, and EPost.
The E-Utilities ESpell (spelling suggestions), EInfo (database statistics),
ECitMatch (batch citation searching in PubMed) and EGQuery (global ESearch) are
currently not supported since they can be either assembled using existing
functions or have a very broad usage. \entrezpy is not primarily intended to
replace an NCBI website search, but to run queries for a specific problem. The
\entrezpy functions implemented use the same parameters as those described in
the Entrez manual. ESearch, ELink and EPost E-Utilities can use the Entrez
History server which includes a reference for the query as part of the result,
consisting of the WebEnv and QueryKey values. Together, these values can be
used in subsequent queries to reference a prior query and are recognized by
\entrezpy.

NCBI limits query and retrieval sizes. For example, downloading summaries in
JSON format is limited to 500 summaries at a time. In such cases, queries must
be split into several requests to obtain the whole requested data set.
\entrezpy automates these steps, enabling the easy assembly of complex
E-Utility queries to search the Entrez databases and download data sets.
Further, Entrez History server responses can be used to link queries, analogous
to piping commands on UNIX systems and the Entrez Direct tool
(Figure~\ref{fig:entrezpy_examples}). \entrezpy is designed to analyze the
response from each request as soon as it is received, allowing the
implementation of checkpoints when handling large data sets, for example
whether to resume after aborts or errors. We implemented multi-threading
support that can be disabled since not all libraries and tools support
multi-threading; that is, the SQLite3 module in the Python standard library
(\url{https://docs.python.org/3.7/library/sqlite3.html#multithreading})
disallows shared connection and cursors between threads.

%-------------------------------------------------------------------------------
%  \author Jan P Buchmann <jan.buchmann@sydney.edu.au>
%  \copyright 2018 The University of Sydney
%  \description
%-------------------------------------------------------------------------------
\newsavebox{\mylistingbox}
\begin{figure}
   \begin{lrbox}{\mylistingbox}%
    \begin{minipage}{\linewidth}%
      \begin{bash}
esearch -db gene -query "tp53[preferred symbol] AND human[orgn]"  \|
elink -target protein  | esummary  \ |
xtract -pattern DocumentSummary -element Caption SourceDb
\end{bash}
    \end{minipage}%
  \end{lrbox}%
  \subfloat[Entrez Direct example to fetch the 'caption' and 'source database'
            information for sequences in the protein database linked from
            results in the gene database]{\usebox{\mylistingbox}\label{subfig:edirect}}%
%next  subfigure
\hfill
  \begin{lrbox}{\mylistingbox}%
    \begin{minipage}{\linewidth}%
      \begin{python}
import esearch.esearcher

p = {'db':'nucleotide',
     'term':'biomol trna',
     'field':'prop'}
es = esearch.esearcher.Esearcher('esearcher', email)
print(es.inquire(parameter=p).result.uids)
\end{python}
    \end{minipage}%
  \end{lrbox}%
  \subfloat[Example of searching the Entrez databases using \entrezpy esearcher.
            The nucleotide Entrez database is queried for UIDs  with the
            term 'biomo trna' in their property field. The found UIDs are
            printed to the standard output using the default EsearcherAnalyzer.]{\usebox{\mylistingbox}\label{subfig:esearch}}%
%next  subfigure
\hfill
  \begin{lrbox}{\mylistingbox}%
    \begin{minipage}{\linewidth}%
      \begin{python}
import wally.wally

p = {'db':'gene', 'term':'tp53[preferred symbol] AND human[organism]'}
w = wally.wally.Wally(email)
px = w.new_pipeline()
qid = px.add_search(parameter=p)
qid = px.add_link(parameter={'db' : 'protein'}, dependency=qid)
qid = px.add_summary(dependency=qid)
analyzer = w.run(px)
for i in analyzer.result.summaries:
  print(analyzer.result.summaries[i].get('caption'),
        analyzer.result.summaries[i].get('sourcedb')))
\end{python}%
    \end{minipage}%
  \end{lrbox}%
  \subfloat[\entrezpy Wally example reproducing Figure~\ref{subfig:edirect}]{\usebox{\mylistingbox}\label{subfig:wally}}%
  \caption{\entrezpy usage examples. Figure~\ref{subfig:edirect} depicts a query
           using the Entrez Direct tool. Figure~\ref{subfig:esearch} shows
           the usage for a single E-Utility function, here ESearch.
           Figure~\ref{subfig:wally} shows the same query as
           Figure~\ref{subfig:edirect} using the Wally class from \entrezpy. The
           email parameter in the \entrezpy examples indicates the email of the
           developer as required by NCBI
  \label{fig:entrezpy_examples}}
\end{figure}


The class Wally simplifies assembling complex queries
(Figure~\ref{fig:entrezpy_examples}). Internally, \entrezpy assigns each query
and request a unique identifier. This allows the creation of query pipelines
using prior queries, thereby limiting re-downloading results to reuse the in a
later step. The versatility of \entrezpy is based on the use of virtual
functions and modular design. We implemented a default analyzer for all
E-Utilities except for EFetch. This is deliberate, since an analyzer for an
EFetch request is usually the last step in query. Given the numerous
possibilities, databases and formats available, this step is best left to the
pipeline developer. Creating a specific analyzer requires the implementation of
only two virtual functions of the \entrezpy analyzer base class, specifically
the methods to handle errors and the result. Therefore, a new and highly
specific analyzer for a specific data set can be written without the need to
adjust the whole request process.

\entrezpy has been designed "to do one thing and do it well". It facilitates
the querying and downloading data from the Entrez databases, one of the largest
life sciences data repositories, while giving a developer the freedom to easily
integrate specific analysis functions.

\section{Funding}
ECH is funded by an ARC Australian Laureate Fellowship (FL170100022).
\bibliography{references}
\bibliographystyle{tpj}
\end{document}
